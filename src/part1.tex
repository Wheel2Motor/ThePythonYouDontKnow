\documentclass{ctexbook}

\usepackage{ctex}

\usepackage[svgnames]{xcolor}
	\definecolor{colorcodeblock}{HTML}{444444}

\usepackage{minted}
	\setminted{
		style=default,
		fontsize=\small,
		bgcolor=WhiteSmoke,
		linenos=true,
		frame=single,
		framesep=1ex,
		showspaces=false,
		showtabs=false,
		xleftmargin=10pt,
		numbersep=5pt,
	}
	\newmintinline[inlinecode]{text}{
		bgcolor=white,
		style=default,
	}

\usepackage{hyperref}

\usepackage{xeCJKfntef}

\usepackage{multicol}
	\setlength{\columnseprule}{0.4pt}


\title{你不知道的Python}
\author{Mr Nobody}

\begin{document}

\begin{titlepage}
\raggedleft
{\scriptsize\itshape \today}\\[3in]
{\Huge\bfseries 你不知道的Python}\\[2mm]
{\large\scshape\underline{The Python You Don't  Know}}\\[5mm]
{\normalsize Author: \itshape Nobody}
\end{titlepage}

\frontmatter

大多数书籍会告诉我们Python如何简单,却少有书籍会告诉我们:\linebreak\textbf{其实Python是一门易学难精的语言}。\\[0.1in]

Python因其语法容易上手、动态类型、内存安全、生态丰富等诸多优点而圈粉无数。
现如今跻身于程序开发圈的子里,似乎身边到处都是会Python的人。

Python的基础语法跟其它编程语言比起来,确实可以说是``新手友好型'',但是其高级特性学起来也是相当抽象。
笔者觉得,C语言的语法比Python更简单,大多数没能学进去C语言的人,不是因为C语言本身难,而是因为C语言假设用户知道计算机是如何工作的;C语言真正的难,在于它设施上的``一穷二白'',以及将计算机工作方式的复杂性直接暴露给了用户。
然而Python并不假设用户知道计算机是如何工作的,所以小学生也可以在经过简单的学习后立刻上手;随着对Python的深入了解,大家会发现Python的闭包、匿名函数、自由变量、装饰器、元素编程等概念相当抽象。

大多数Python开发者只用到了Python最基础的功能,用Python的语法写另一种C、C++,并没能有效发挥出Python的优势。
Python的语法提供了很多``语法糖'',如多值、匿名函数、解包、元编程等;Python的标准库提供了相当多的常用功能,被大家称为``自带电池''\marginpar{自带电池:self operated},意思是说只用其标准库就可以做很多事情。
善用Python的语法特性和标准库,能够有效帮助我们把代码写得更快更优雅。

很多人说Python的性能不行,Python的易用性确实是以性能为代价的,但现代的Python生态大多只是借用了Python语言作``外壳'',底层性能悠关的环节实际运行的是C、C++、Fortran等性能更好的语言编官写的库。
语言本身的速度确实没法跟编译型语言媲美,但是也没有外界传的相差几百几千倍。
笔者在实践中做过很多次将现有的Python算法移植到C语言,结果发现时间复杂度为$O(n)$的算法,单核性能会得到大约50--100倍左右的提升。
优化得好的Python代码,会在内部运行其C语言的内联实现\footnote{CPython是这样的,其它实现要看情况了},性能直追C++。
当然,GIL\footnote{Global Interpreter Lock:全局解释器锁,会导致Python的多线程只能跑在单核上,无法发恢出多核CPU的优势。}确实是一个较大的性能束缚,幸运的是,从Python3.13开始,已经能够以no-gil模式运行了,可以实现多核多线程。

Python的定位就是鼓励大家善用现有的设施,好好做个调包侠,没有什么可耻的,毕竟——\textbf{人生苦短,我用Python}。\\[0.1in]

同样是Python3,编写本书时的3.12相比3.5已经产生了翻天覆地的变化,Python本身也在跟着时代进步。
本书中Python代码案例采用Python3.12进行测试,如有版本上的要求,会特别指出。

这并不是一本正而八经编写的的书籍,只是是笔者对工作中踩过的坑进行的记录。
本书的内容假设您有一定的Python开发经验,如果您有一定的C、C++、C\#、Java语言开发经验会更有助于理解。

笔者才疏学浅,如果您在阅读的过程中发现有错误的地方,感谢您的指正:leepypy@qq.com。

\tableofcontents

\part{改善代码的诸多方法}

\chapter{语法糖}
很多习惯于静态语言的程序员会在心中觉得,语法糖是奇技淫巧而已,无需当回事。
但语法糖确实能够有效帮助人们避免给代码写得一团糟。

\section{多值与解包}

\subsection{多值}

\subsubsection{多值}\marginpar{多值:multi-value}

在传统的编程语言如C语言中,函数只能返回一个结果。
比如一个数的平方根有正负两个解,但C语言中的函数只能返回一个结果。
要解决这个问题,要不返回一个结构体;要不增加两个指针作为入参。

\begin{minted}{c}
typedef struct _Result {
    float pos, neg;
} Result;

Result MySqrt1(float num) {
    Result res;
    res.pos = sqrt(num);
    res.neg = -res.pos;
    return Result;
}

void MySqrt2(float num, float* pos, float* neg) {
    *pos = sqrt(num);
    *neg = -(*pos);
}
\end{minted}

但Python的函数能返回任意数量的结果,函数能返回多个结果的这个功能就叫做``多值''。

\begin{minted}{py}
def mysqrt(num):
    pos = math.sqrt(num)
    neg = -pos
    return pos, neg
\end{minted}

\subsection{解包}\marginpar{解包:unpacking}

\subsubsection{多值函数的返回值是什么}

要了解什么是解包,还要先了解一个多值函数的返回值到底是什么。
返回多值的函数,其返回值可以在同一个赋值语句中被多个变量接收。

\begin{minted}{py}
pos, neg = mysqrt(4)
\end{minted}

但是如果其返回值只赋值到一个变量上,得到的是一个元组。

\begin{minted}{py}
res = mysqrt(4) # (2, -2)
\end{minted}

这并不是什么隐式转换,而是Python的表达式语法,逗号分隔的连续多个子表达式构成元组的表达式,
括号不是必须的{\scriptsize(空元组除外)}。
\marginpar{\scriptsize\hrule\vspace{1ex}官方文档中有提到:Note that tuples are
not formed by the parentheses , but rather by use of the comma operator. The
exception is the empty tuple, for which parentheses are required — allowing
unparenthesized “nothing” in expressions would cause ambiguities and allow
common typos to pass uncaught.}
所以返回多值的函数,本质上是返回了一个元组。
我们平时写元组都加括号,主要还是为了避免和其它语法结构产生冲突,比如放在函数的形参列表中,
可能会被认为是函数的多个参数。

\begin{minted}{py}
1, 1+1, 4-1 # (1, 2, 3)
\end{minted}

\subsubsection{什么是解包}

Python中的可迭代对象
\marginpar{可迭代对象:iterable object}
\footnote{Python中的可迭代对象都是collections.abc.Iterable的子类或虚子类,它们都实现了\inlinecode{__iter__}方法,
可以用\inlinecode{for ... in}语法进行迭代,但是未必都有长度,比如一个可无限迭代的生成器。
拥有长度的对象都是collections.abc.Sized的子类或虚子类,可用len()函数获取长度。
list和tuple就同时是Iterable和Sized的虚子类。
虚子类是指其符合基类的必要协议,但并不直接集成自基类,这种常见于CPython的C语言内置对象。}
包括字符串、列表、元组、集合、字典、生成器等。
所有能用\inlinecode{for ... in}语法进行迭代的对象都是可迭代对象。

可迭代对象可以展开,在赋值语句中同时赋值给多个变量,或变成另一个可迭代对象表达式上下文中的多个元素,这就叫做解包。

\begin{minted}{py}
a, b = 1, 2
c, d = (1, 2)
e, f = [1, 2]
g, h = "12"
tup = (3, 4)
lst = [1, 2, *tup] # [1, 2, 3, 4]
\end{minted}

\subsubsection{可迭代对象解包}
可迭代对象在赋值语句中同时对多个变量赋值,只需在赋值号左边提供的变量数量和可迭代对象的长度一致即可,
变量名之间用逗号隔开。

\marginpar{\scriptsize\hrule\vspace{1ex}在for \ldots{} in语法中,迭代变量就是负责接收每次迭代时返回的结果的变量。
可以理解为,每次迭代返回的值都被赋值给了迭代变量。
如果每次迭代返回的值也是可迭代对象,那么当提供了多个迭代变量,也就被自然而然地解包给了每一个。}

\begin{minted}{py}
pos = [1, 2]
x, y = pos
for name, age in [("小明", 24), ("小强", 25)]:
    print(f"{name}'s age is {age}")
\end{minted}

还可以利用解包的功能,轻松实现交换两个变量。
其本质原理就是构造了一个新的元组,然后再按顺序解包给变量。

\begin{minted}{py}
a, b = 1, 2
b, a = a, b
\end{minted}

其实赋值号左边提供的变量数量和可迭代对象的长度不一致也可以。
可以回想一个Python的函数定义中,当需要按顺序提供不确定数量的参数时的做法:星号。
星号装饰过的变量,只能放在解包变量的尾部,它会在赋值时,``贪婪''地将后续的一切内容包裹到一个元组中。

\marginpar{\scriptsize\hrule\vspace{1ex}接收数据的变量数量和解包的可迭代对象长度不一致的情况,可以考虑有这么一行数据,
对用户有用的只有前两个,为剩下的数据起名实在是一件很头痛的事,毕竟起名实在是太难了。}

\begin{minted}{py}
info = [
    ("李华", 25, 165.5, 55.3),
    ("小刚", 26, 170.1, 65.0),
    ("小明", 23, 172.0, 67.2),
]
for name, age, *_ in info: # 不知道起什么名称时,下划线是不错的选择
    print(f"{name}'s age is {age}")
\end{minted}

可迭代对象还可以解包为另一个可迭代对象中的多个元素,也是通过星号。
值得注意的是函数的形式参数列表也可以算是一种可迭代对象
\footnote{CPython的C代码中,对函数形式参数列表的访问,就是从元组对象身上通过\inlinecode{PyArg_ParseTuple}实现的,
CPython给函数对象的参数列表封装成了一个元组。}
,所以也可以在参数列表中展开。
\marginpar{\scriptsize\hrule\vspace{1ex}官方文档中关于单个星号解包语法的简短描述:An asterisk * denotes iterable
unpacking. Its operand must be an iterable. The iterable is expanded into a sequence of items, which are included in
the new tuple, list, or set, at the site of the unpacking.}

\begin{minted}{py}
# 在另一个可迭代对象中解包
lst1 = [1, 2, 3]
lst2 = [*lst1, 4, 5, 6] # [1, 2, 3, 4, 5, 6]
# 在函数参数列表中解包
def dist(x1, y1, x2, y2):
    xdiff = x2 - x1
    ydiff = y2 - y1
    return sqrt(xdiff ** 2 + ydiff ** 2)
pos1 = (1, 2)
pos2 = (3, 4)
dist(*pos1, *pos2)
\end{minted}

对可迭代对象的解包顺序取决于可迭代对象本身迭代的顺序。
对列表、元组、字符串这类有序对象,展开的顺序就是其自身元素储存的顺序;对集合、字典这类无序对象,
因为其本身就是无序数据结构,迭代的顺序是无法保证的,因此展开后的元素顺序也无法保证。
且字典的解包只能解包出来其键,因为字典每次迭代确实是只返回一个键。

对生成器展开后的顺序和生成器求值得到数据的顺序是一致的。
如果是一个无限生成器,那么程序会卡死,因为解包的过程中会对可迭代对象不断迭代,直到迭代结束。

\marginpar{\scriptsize\hrule\vspace{1ex}(i for i in range(3))是生成器初始式,并不是元组初始式,
Python中并没有元组初始式。}

\begin{minted}{py}
gen1 = (i for i in range(3))
[*gen1, 3, 4, 5] # [0, 1, 2, 3, 4, 5]

def gen2():
    i = 0
    while True:
        yield i
        i += 1
[*gen2, 3, 4, 5] # 程序会卡死
\end{minted}

\begin{multicols}{2}

\begin{minted}[linenos=false]{py}
a = 1,
b = 2
c = 1 + 2
\end{minted}

解包特性产生的bug也会极其难以排察。
代码量大的情况下很难发现这个案例中,``1''的后面有一个``,'',它会导致后续的代码中报出类型错误。
乍一看这跟解包没啥关系,但这种bug多是由于代码编写的过程中有一段时期利用过解包特性,因此\inlinecode{a}曾经是一个元组,后期不再使用解包特性时,删除代码漏删了逗号。

\end{multicols}

\subsubsection{字典解包}
字典也是前面介绍的可迭代对象中的一种,但它有点特殊,所以拿出来单独讲。
严格来说它是一种映射对象
\marginpar{映射对象:mapping object}
\footnote{Python中的映射对象都是\inlinecode{collections.abc.Mapping}的子类或虚子类。}
,同时映射对象也是可迭代对象的子类。

前面说过字典的解包,只能得到其键,不能得到其值,因为字典迭代的时候确实每次只返回一个键。
但有一种我们无时无刻不在用的字点键值对解包方法容易被忽视:\inlinecode{for k, v in yourdict.items()}。
因为字典的\inlinecode{items}方法会返回一个\inlinecode{dict_items}可迭代对象,其每次迭代后返回一个元组,
元组的内容正是一对键值对,此时迭代变量有两个,正好将元组内的键和值解包。

字典还有一种专用的解包语法:两个星号解包键值对。
不过这个语法不能随便用,它有两个要求:字典的键都是字符串,只能在函数的形参列表中使用。它相当于是用字典的键去
匹配函数的参数名称并将值输入进去。

\marginpar{\scriptsize\hrule\vspace{1ex}官方文档中关于两个星号解包语法的简短描:A double asterisk ** denotes dictionary
unpacking. Its operand must be a mapping. Each mapping item is added to the new dictionary. Later values replace values
already set by earlier key/datum pairs and earlier dictionary unpackings.}

\begin{minted}{py}
info = [
    {"name": "李华", "age":25, "height":165.5, "weight":55.3},
    {"name": "小刚", "age":26, "height":170.1, "weight":65.0},
]

def intro(name, age, height, weight):
    print(f"{name}'s age is {age}, "
          f"height - {height}, "
          f"weight - {weight}")
for item in info:
    intro(**item)
\end{minted}

至此可能还是不会觉得有什么用,但是试想你有一个json字典,反序列化后直接就可以塞入函数中使用。

\begin{minted}{py}
allinfo = json.load(your_json_file)
for info in allinfo:
    intro(**info)
\end{minted}

\section{闭包}

\subsection{什么是闭包}

有时候某个函数中会临时用到一个临时函数,它只对当前作用域有意义,脱离当前环境后不会再用上。

\marginpar{\scriptsize\hrule\vspace{1ex}在《C Primer Plus》中关于翻译单元的描述:一个源代码文件和它所包含的头文件。大体就是指,预处理完成后、编译前这个时期的产物。}

在C语言中,我们通常定义一个static函数来限制该函数不会在编译后导出符号
\footnote{C语言默认情况下函数或全局变量如果没写static或extern修饰符,都是extern来修饰。
如果被extern修饰过,那么编译后函数、全局变量的名称就具有全局链接属性,可以和其它库进行链接;如果被static修饰过,就具有静态链接属性,只能在当前翻译单元内使用。
保留函数、全局变量名称到二进制数据中的这个行为就叫做导出符号。
如果两个目标文件中拥有同名符号,那么链接时就会产生符号重定义的错误。}
,这样即使其它翻译单元也有一个同名函数,但是在链接时并不会产生冲突。

\begin{minted}{c}
static max_of_3(int a, int b, int c) {
    int ret = a;
    if (b > ret) ret = b;
    if (c > ret) ret = c;
    return ret;
}
\end{minted}

但这样做还是会有漏洞,万一被该翻译单元内的其它函数误用了怎么办。
通常为了方便后续维护,会在代码中写上详细的使用说明。

Python运行时的对象具有一切皆数据的特性。
无论普通的数据类型如整数、字符串,还是函数和类,它们都是数据,没有指令和数据的区分,都可以在程序运行过程中动态生成。
所以我们可以在函数中定义函数,甚至在函数中定义类,这种行为叫做闭包。

在Python中我们可以直接在一个函数内部定义另一个函数,这样在内部定义的函数就成了其所在函数中的一个局部变量,确保无法在更广阔的外部作用域中访问到,也不容易对后续的开发造成干扰。

\begin{minted}{py}
def max_of_tup3s(tup3s):
    def max_of_3(a, b, c):
       ret = a
       if b > ret: ret = b
       if c > ret: ret = c
       return ret
    return [max_of_3(*tup3) for tup3 in tup3s]

tup3s = ((1, 2, 3), (4, 5, 6))
max_of_tup3s(tup3s) # (3, 6)
\end{minted}

这种在函数中定义函数来进行功能隔离的行为,就叫做闭包。

\subsection{循环体内的闭包函数会被多次编译吗}

由于Python中的函数可以在程序运行中被动态生成,所以如果闭包函数定义被写在了循环体内,会导致反复生成函数对象。
话说到这可能着急的读者已经开始改代码了,但其实大多数情况下根本不用着急。

\begin{minted}{py}
def outter():
    for i in range(100):
        def pow2(n): return n * n
        pow2(i)
\end{minted}

可能一些读者曾经在各种书籍、视频教程中看到这样一种描述——``Python是一行一行边解释边执行的''。
其实这么说并不准确,Python是一种JIT语言\footnote{Just In Time语言:TODO},它有自己``编译''的过程。

Python实际执行的是其字节码。
解释器可以在导入模块时对整个模块进行编译,并在本地写入其编译得到的字节码供后续使用,这样下次就不用再次编译。
也可以在运行过程中将一段字符串编译为字节码,然后立即执行。

我们可以用标准库的\inlinecode{dis}模块中的\inlinecode{dis}方法对一个函数对象``反编译'',看到函数编译后的字节码。

\begin{minted}{py}
import dis
dis.dis(outter)
# 1     0 RESUME         0
# 
# 2     2 LOAD_GLOBAL    1 (NULL + range)
#      12 LOAD_CONST     1 (100)
#      14 CALL           1
#      22 GET_ITER
#   >> 24 FOR_ITER      13 (to 54)
#      28 STORE_FAST     0 (i)
# 
# 3    30 LOAD_CONST     2 (<code object pow2 at 0x102d4bab0>)
#      32 MAKE_FUNCTION  0
#      34 STORE_FAST     1 (pow2)
# 
# 4    36 PUSH_NULL
#      38 LOAD_FAST      1 (pow2)
#      40 LOAD_FAST      0 (i)
#      42 CALL           1
#      50 POP_TOP
#      52 JUMP_BACKWARD 15 (to 24)
# 
# 2 >> 54 END_FOR
#      56 RETURN_CONST   0 (None)
# 
# Disassembly of <code object pow2 at 0x102d4bab0>:
# 3     0 RESUME         0
#       2 LOAD_FAST      0 (n)
#       4 LOAD_FAST      0 (n)
#       6 BINARY_OP      5 (*)
#      10 RETURN_VALUE
\end{minted}

通过对字节码的观察,我们不难发现,函数\verb|pow2|的字节码只产生了一次编译结果。
但是解释器会在循环中多次通过指令\verb|LOAD_CONST|加载编译好的字节码,然后通过指令\verb|MAKE_FUNCTION|产生一个新的函数对象。

虽然\verb|LOAD_CONST|和\verb|MAKE_FUNCTION|会被冗余执行,但是Python的循环语句本身就比较慢,如果循环本身次数不那么多,每次循环体内多这两个指令也无伤大雅,但是如果循环的次数多了,叠加效应就会变得显著。

\textbf{但无论如何,这都不是一个好的编程习惯。}

\subsection{捕获与自由变量}

闭包函数能够访问其外部函数的局部变量,且在其外部函数执行完成后,仍然能够继续访问该变量。

\begin{minted}{py}
def return_a_number(n):
    num = n
    def impl():
        return num
    return impl
\end{minted}

闭包函数访问外部函数作用域内的局部变量,这种行为叫作捕获;被闭包函数访问的外部函数作用域内的变量叫自由变量,在一些编程语言中也叫上值。\marginpar{上值:up-value}

由于自由变量本身也是外部函数的局部变量,所以外部函数是可重入\marginpar{可重入:re-entrant\\是指不同时序下多次调用互不影响。}的。
外部函数多次执行返回的闭包函数,即使在代码里访问的都是同一个自由变量,但它们都是外部函数执行时的局部变量,所以互不影响。

\begin{minted}{py}
r3 = return_a_number(3)
r4 = return_a_number(4)
r3() # 3
r4() # 4
\end{minted}

Python中为变量赋值,由于不需要声明,所以在函数中为变量赋值时,到底是为全局变量、自由变量还是局部变量赋值就容易产生歧意。\marginpar{\scriptsize\hrule\vspace{1ex}函数内变量只是读取而不写入的话就不会产生歧意,只有需要赋值时才会产生歧意,不声明global或nonlocal的话,默认就是局部变量。}
为了消除歧意,如果要为全局变量赋值,要在赋值前使用\verb|global|命令在函数内声明全局变量;如果要为自由变量赋值,要在赋值前使用\verb|nonlocal|命令在闭包函数内声明自由变量

\begin{minted}{py}
total_counter = 0
def counter(n):
    closure_counter = 0
    def impl():
        global total_counter
        nonlocal closure_counter
        total_counter += 1
        closure_counter += 1
        return closure_counter, total_counter
    return impl

c1 = counter()
c2 = counter()
c1() # 1, 1
c1() # 2, 2
c2() # 1, 3
c2() # 2, 4
\end{minted}

\section{匿名函数}

有时候需要临时用到一个简单函数的时候,定义一个闭包函数也觉得大动干戈了,还可以定义一个匿名函数,直接到一个变量上。

匿名函数形式:\verb|lambda <arg1>, <arg2>, <...>: <expression>|

使用\verb|lambda|关键字进行定义,冒号分隔参数列表和函数体。
\marginpar{\scriptsize\hrule\vspace{1ex}Python中的函数即使没使用return语句,也会默认返回None。我们可以用这个特性,使用or结构将不返回内容的函数结合起来,形成链式调用。}
参数列表可以没有参数或任意数量的参数,使用逗号分隔,参数列表前后不需要括号。
函数体是一个表达式,所以也意味着其必然得有返回值,即使是返回None。

\section{高阶函数}

	\subsection{map}
	\subsection{filter}
	\subsection{reduce}

\section{装饰器}

	\subsection{装饰器有什么用}
	\subsection{自定义装饰器}
	\subsection{自定义带参数的装饰器}
	\subsection{手动实现property装饰器}

\part{类}

	\section{魔法方法}

\part{模块与包}

	\chapter{模块与包各自怎么理解}
		\section{模块是什么}
		\section{包是什么}
		\section{相对导入}

	\chapter{导入模块}
		\section{\_\_import\_\_}
		\section{importlib}
		\section{重新导入模块}
		\section{动态导入模块}

\part{标准库的设施}

	\chapter{数据结构}
	\chapter{路径操作}
	\chapter{枚举}
	\chapter{数据类}


\part{高级特性}
	\chapter{元类}
	\chapter{描述符协议}
	\chapter{类型注解}


\part{高性能编程}
\chapter{与C语言混合编程}
\chapter{numpy}
\chapter{cython}


\end{document}
